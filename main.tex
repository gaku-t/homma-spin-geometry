\documentclass[dvipdfmx]{amsart}

\usepackage{amsmath, amssymb}
\usepackage{tikz-cd}

\theoremstyle{definition}
\newtheorem{ans}{}
\numberwithin{ans}{section}

\newcommand{\fakesection}[1]{%
  \par\refstepcounter{section}% Increase section counter
  \sectionmark{#1}% Add section mark (header)
  \addcontentsline{toc}{section}{\protect\numberline{\thesection}#1}% Add section to ToC
  % Add more content here, if needed.
}

\newcommand{\Hom}[2]{\mathrm{Hom}(#1, #2)}


\begin{document}

\fakesection{第1章 クリフォード代数}

\begin{ans}
  $((c_1v_1 + c_2v_2) \otimes w)(B)
  = B(c_1v_1 + c_2v_2, w)
  = c_1B(v_1, w) + c_2B(v_2, w)
  = c_1(v_1 \otimes w)(B) + c_2(v_2 \otimes w)(B)
  = (c_1(v_1 \otimes w) + c_2(v_2 \otimes w))(B)
  $.
  もう一方の式も同様.
\end{ans}

\begin{ans}
  $F \times G(c_1v_1 + c_2v_2, w)
  = F(c_1v_1 + c_2v_2) \otimes G(w)
  = (c_1F(v_1) + c_2F(v_2)) \otimes G(w)
  = c_1F(v_1) \otimes G(w) + c_2F(v_2) \otimes G(w)
  = c_1F \times G(v_1, w) + c_2F \times G(v_2, w)
  $. よって第1引数について線形である. 第2引数についての線形性も同様に示せる.
\end{ans}

\begin{ans}
  写像$B: V^\ast \times W \rightarrow \Hom{V}{W}$を,
  $(v^\ast, w) \mapsto (\varphi: v \mapsto v^\ast(v)w)$と定義する.
  これは双線形なので, 次の図式を可換とするような線形写像$\beta$が存在する.
  \[
    \begin{tikzcd}
      V^\ast \times W \ar[r, "\otimes"] \arrow[rd, "B"'] & V^\ast \otimes W \ar[d, "\beta"] \\
      & \Hom{V}{W}
    \end{tikzcd}
  \]
  $\beta$が同型であることを示したいが,
  $V^\ast \otimes W$と$\Hom{V}{W}$はともに次元が$\dim V \cdot \dim W$なので,
  $\beta$が全射であることを見ればよい.
  そのためには, $V$の基底$\{e_1,.., e_n\}$と$W$の基底$\{f_1,..., f_m\}$を適当にとり,
  $1 \le \forall i \le n$, $1 \le \forall j \le m$について$\varphi_{ij}(e_k) = \delta_{ik}f_j\ (k = 1,..., n)$
  をみたすような$\varphi_{ij}$ ($\Hom{V}{W}$の基底) が$\beta$の像に入っていることをみればよい.
  このことは, $V^\ast$の双対基底$\{e^\ast_1,..., e^\ast_n\}$について
  $B(e^\ast_i, f_j)(e_k) = e^\ast_i(e_k)f_j = \delta_{ik}f_j = \varphi_{ij}(e_k)$であることから分かる.
\end{ans}

\end{document}
