\documentclass{amsart}

\usepackage{amsmath, amssymb}

\theoremstyle{definition}
\newtheorem{ans}{}
\numberwithin{ans}{section}

\newcommand{\fakesection}[1]{%
  \par\refstepcounter{section}% Increase section counter
  \sectionmark{#1}% Add section mark (header)
  \addcontentsline{toc}{section}{\protect\numberline{\thesection}#1}% Add section to ToC
  % Add more content here, if needed.
}


\begin{document}

\fakesection{第1章 クリフォード代数}

\begin{ans}
  $((c_1v_1 + c_2v_2) \otimes w)(B)
  = B(c_1v_1 + c_2v_2, w)
  = c_1B(v_1, w) + c_2B(v_2, w)
  = c_1(v_1 \otimes w)(B) + c_2(v_2 \otimes w)(B)
  = (c_1(v_1 \otimes w) + c_2(v_2 \otimes w))(B)
  $.
  もう一方の式も同様.
\end{ans}

\begin{ans}
  $F \times G(c_1v_1 + c_2v_2, w)
  = F(c_1v_1 + c_2v_2) \otimes G(w)
  = (c_1F(v_1) + c_2F(v_2)) \otimes G(w)
  = c_1F(v_1) \otimes G(w) + c_2F(v_2) \otimes G(w)
  = c_1F \times G(v_1, w) + c_2F \times G(v_2, w)
  $. よって第1引数について線形である. 第2引数についての線形性も同様に示せる.
\end{ans}

\end{document}
