\documentclass[dvipdfmx]{amsart}

\usepackage{amsmath, amssymb}
\usepackage{tikz-cd}
\usepackage{framed}

\theoremstyle{definition}
\newtheorem{ans}{}
\numberwithin{ans}{section}

\newtheorem{mynote}{行間メモ}
\numberwithin{mynote}{section}
\newenvironment{note}
  {\begin{leftbar}\begin{mynote}}
  {\end{mynote}\end{leftbar}}

\newcommand{\fakesection}[1]{%
  \par\refstepcounter{section}% Increase section counter
  \sectionmark{#1}% Add section mark (header)
  \addcontentsline{toc}{section}{\protect\numberline{\thesection}#1}% Add section to ToC
  % Add more content here, if needed.
}

\DeclareMathOperator*{\sgn}{sgn}

\newcommand{\Hom}[2]{\mathrm{Hom}(#1, #2)}


\begin{document}

\fakesection{第1章 クリフォード代数}

\begin{ans}
  $((c_1v_1 + c_2v_2) \otimes w)(B)
  = B(c_1v_1 + c_2v_2, w)
  = c_1B(v_1, w) + c_2B(v_2, w)
  = c_1(v_1 \otimes w)(B) + c_2(v_2 \otimes w)(B)
  = (c_1(v_1 \otimes w) + c_2(v_2 \otimes w))(B)
  $.
  もう一方の式も同様.
\end{ans}

\begin{ans}
  $F \times G(c_1v_1 + c_2v_2, w)
  = F(c_1v_1 + c_2v_2) \otimes G(w)
  = (c_1F(v_1) + c_2F(v_2)) \otimes G(w)
  = c_1F(v_1) \otimes G(w) + c_2F(v_2) \otimes G(w)
  = c_1F \times G(v_1, w) + c_2F \times G(v_2, w)
  $. よって第1引数について線形である. 第2引数についての線形性も同様に示せる.
\end{ans}

\begin{ans}
  写像$B: V^\ast \times W \rightarrow \Hom{V}{W}$を,
  $(v^\ast, w) \mapsto (\varphi: v \mapsto v^\ast(v)w)$と定義する.
  これは双線形なので, 次の図式を可換とするような線形写像$\beta$が存在する.
  \[
    \begin{tikzcd}
      V^\ast \times W \ar[r, "\otimes"] \arrow[rd, "B"'] & V^\ast \otimes W \ar[d, "\beta"] \\
      & \Hom{V}{W}
    \end{tikzcd}
  \]
  $\beta$が同型であることを示したいが,
  $V^\ast \otimes W$と$\Hom{V}{W}$はともに次元が$\dim V \cdot \dim W$なので,
  $\beta$が全射であることを見ればよい.
  そのためには, $V$の基底$\{e_1,.., e_n\}$と$W$の基底$\{f_1,..., f_m\}$を適当にとり,
  $1 \le \forall i \le n$, $1 \le \forall j \le m$について$\varphi_{ij}(e_k) = \delta_{ik}f_j\ (k = 1,..., n)$
  をみたすような$\varphi_{ij}$ ($\Hom{V}{W}$の基底) が$\beta$の像に入っていることをみればよい.
  このことは, $V^\ast$の双対基底$\{e^\ast_1,..., e^\ast_n\}$について
  $B(e^\ast_i, f_j)(e_k) = e^\ast_i(e_k)f_j = \delta_{ik}f_j = \varphi_{ij}(e_k)$であることから分かる.
\end{ans}

\begin{ans}
  (1) $(\bigwedge^p(V))^\ast \ni \alpha$に対して,
  $\alpha \circ \pi \circ \otimes^p$は$p$次交代形式である.
  逆に, $p$次交代形式$A$に対して, $\alpha \in (\bigwedge^p(V))^\ast$を,
  $\alpha(v_1 \wedge \cdots \wedge v_p) = A(v_1,... v_p)$を線形に拡張することで定義できる.
  この$\alpha$がwell-definedであることを確かめる必要があるが,
  まず$A$が$p$次線形形式であることから, $\bigotimes^pV$上のある線形形式$\beta$により
  $\beta(v_1 \otimes \cdots \otimes v_p) = A(v_1,... v_p)$と書ける.
  $A$の交代性から, $\beta$は$I(V) \cap \bigotimes^pV$上で$0$である. したがって, $\beta$は$\alpha$を誘導する.
  このように定義した$\alpha$に対して$\alpha \circ \pi \circ \otimes^p = A$が成り立つから,
  この対応により$(\bigwedge^p(V))^\ast$と$p$次交代形式全体を同一視できる.\\
  (2) 前小問で見たことと同様であるが, $A$は$p$次線形写像なので,
  線形写像$\beta: \bigotimes^pV \rightarrow U$で$A = \beta \circ \otimes^p$を満たすものがただ$1$つ存在する.
  $A$の交代性から$\beta$は$I(V) \cap \bigotimes^pV$上で$0$なので,
  $\alpha: \bigwedge^pV \rightarrow U$を誘導し, $\alpha \circ \wedge^p = A$である.
  $\beta$の一意性から, これを満たす$\alpha$は一意である.
  \[
    \begin{tikzcd}
      V \times \cdots \times V \ar[r, "\otimes^p"] \ar[rd, "A"'] &\bigotimes^pV \ar[d, "\beta"'] \ar[r, "\pi"] & \bigwedge^pV \ar[ld, dashed, "\alpha"]\\
      & U &
    \end{tikzcd}
  \]
  \\
  (3) テンソル代数の普遍性から, 代数準同型$F_T: T^\ast(V) \rightarrow A$で$F_T|_V = f$となるものがただ$1$つ存在する.
  $F_T$はイデアル$I(V)$上で$0$なので, 準同型$F: \bigwedge^\ast(V) \rightarrow A$を誘導する.
  $F$の一意性は, $F_T$の一意性からしたがう.
\end{ans}

\begin{ans}
  基底であることはすでに示してあったので, 内積を調べる.
  $\langle e_{i_1} \wedge \cdots \wedge e_{i_p}, e_{i_1} \wedge \cdots \wedge e_{i_p} \rangle
  = \det(\delta_{ij})_{ij} = 1$.
  また, $e_{i_1} \wedge \cdots \wedge e_{i_p} \neq e_{i^\prime_1} \wedge \cdots \wedge e_{i^\prime_p}\ (1 \le i_1 < \cdots < i_p \le n,\ 1 \le i^\prime_1 < \cdots < i^\prime_p \le n)$
  ならば, ある$i_j$について$i_j \notin \{i^\prime_1,..., i^\prime_p\}$である.
  よって行列$(\langle e_{i_k}, e_{i^\prime_l} \rangle)_{kl}$は第$j$行の成分がすべて$0$となり, 行列式は$0$である. よって,
  $\langle e_{i_1} \wedge \cdots \wedge e_{i_p} , e_{i^\prime_1} \wedge \cdots \wedge e_{i^\prime_p} \rangle = 0$.
\end{ans}

\begin{ans}
  別の正の正規直交基底$(e^\prime_1,.., e^\prime_n)$をとると,
  $\phi = e_1 \wedge \cdots \wedge e_n$と$\psi = e^\prime_1 \wedge \cdots \wedge e^\prime_n$
  がともに正の向きに入っているので, $\psi = \alpha \phi\ (\alpha > 0)$と書ける.
  $1 = \langle \psi, \psi \rangle = \langle \alpha \phi, \alpha \phi \rangle = \alpha^2$
  より$\alpha = 1$なので, $\phi = \psi$.
\end{ans}

\begin{note}
  定義1.6がwell-definedであるかどうかは次の問で確かめることになるが,
  今のところは正規直交基底$e_1,...,e_n$を$1$つ固定していると考えればよい.
  この基底について定義1.6の等式が成り立っているとき,
  基底の任意の置換$e_{\sigma(1)},...,e_{\sigma(n)}$についても
  等式が成り立っているとしても矛盾しないので、それも定義のうちに入れておく.
  すると、$\bigwedge^\ast(V)$の基底での$\ast$の値が与えられたことになるので,
  $\ast$を線形に$\bigwedge^\ast(V)$上に拡張できる.
\end{note}

\begin{ans}
  \underline{\textbf{前半}}: $e_{p+1},...,e_n,e_1,...,e_p$の符号は,
  $e_1,...,e_n$の符号に$(-1)^{p(n-p)}$を掛けたものである.
  (バブルソートを行なっていると考えれば, $1,..,p$を昇順の位置に移動させるのにそれぞれ$n-p$回の互換を行なうことになる.)
  よって, $\ast^2(e_1 \wedge \cdots \wedge e_p)
  = \ast(\pm e_{p+1} \wedge \cdots \wedge e_n)
  = \pm \cdot \pm (-1)^{p(n-p)}e_1 \wedge \cdots \wedge e_p
  = (-1)^{p(n-p)}e_1 \wedge \cdots \wedge e_p
  $. 他の基底についても同様であるが, 置換の符号が$2$回現れてキャンセルすることになる.\\
  \underline{\textbf{後半}}: $\ast$の定義に用いた$V$の正規直交基底を$e_1,...,e_n$とする.
  ウェッジ積, 内積, および$\ast$の (双) 線形性から,
  $\phi = e_{i_1} \wedge \cdots \wedge e_{i_p}\ (i_1 < \cdots < i_p)$,
  $\psi = e_{j_1} \wedge \cdots \wedge e_{j_p}\ (j_1 < \cdots < j_p)$
  の場合にのみ等式を示せばよい.
  これらについて等式が成り立つことは, $\phi \neq \psi$の場合は, すべて$0$となるので明らかであろう.
  そこで$\phi = \psi$の場合を考える.
  適当な置換$\sigma$によって$\phi = e_{\sigma(1)} \wedge \cdots \wedge e_{\sigma(p)}$と書けるとする.
  $\phi \wedge \ast \phi
  = (e_{\sigma(1)} \wedge \cdots \wedge e_{\sigma(p)}) \wedge (\pm \sgn \sigma e_{\sigma(p + 1)} \wedge \cdots \wedge e_{\sigma(n)})
  = \pm \sgn \sigma e_{\sigma(1)} \wedge \cdots \wedge e_{\sigma(n)}
  = \pm e_1 \wedge \cdots \wedge e_n
  = \mathrm{vol}
  $. ただしここで$\pm$は基底$e_1,...,e_n$の正負に対応している. このことから, 示すべきすべての等式が成り立つことは明らかであろう.
  さて, ここで得られた関係式から, $\ast$がwell-definedであることを確認しておく.
  別の基底から$\ast^\prime$を定義したとすると,
  任意の$\phi \in \bigwedge^p(V), \psi \in \bigwedge^{n-p}(V)$に対して,
  $\langle \phi, \ast \psi \rangle \mathrm{vol}
  = \phi \wedge \ast^2 \psi
  = (-1)^{p(n-p)}\phi \wedge \psi
  = \phi \wedge {\ast^\prime}^2 \psi
  = \langle \phi, \ast^\prime \psi \rangle \mathrm{vol}$.
  したがって$\langle \phi, \ast \psi \rangle = \langle \phi, \ast^\prime \psi \rangle$
  であるから, $\ast = \ast^\prime$である.
\end{ans}

\begin{note}
  命題1.8の補足. $V$に正規直交基底$e_1,..,e_n$が与えられているとき,
  線形写像$f$が関係式を満たすことを確認するには,
  $f(e_i)f(e_j) + f(e_j)f(e_i) = 2\delta_{ij}$
  であることを確かめればよい.
  逆にこれが成り立つように$f(e_i)$の値を定めておけば, それを$V$上に線形に拡張した
  $f: V \rightarrow A$が関係式を満たす.
  このことは例1.5, 例1.6では明示されていないが, 後の補題1.1の証明に用いられている.
\end{note}

\begin{note}
  例1.5について. 与えられた四元数関係式から, $ji = (ki)(jk) = k^3 = -k = -ij$などもしたがう.
  このことから四元数共役について$\overline{pq} = \overline{q}\ \overline{p}$も確かめられる.
\end{note}

\begin{ans}
  $f(e_i)f(e_j) + f(e_j)f(e_i) = 2\delta_{ij}$を満たすので,
  これを$\mathbb{R}^4$上に線形に拡張すると,
  線形写像$f: \mathbb{R}^4 \rightarrow \mathbb{H}(2)$は$f(v)^2 = -\langle v, v \rangle 1$を満たす.
  したがって代数準同型$\mathrm{Cl}_{4, 0} \rightarrow \mathbb{H}(2)$がある.
  $\mathbb{R}$ベクトル空間としての次元はいずれも$16$なので, この準同型が全射であることを確かめればよい.
  全射であることについては,
  $f(e_1)^4 = \begin{pmatrix}
    1 & 0 \\
    0 & 1
  \end{pmatrix}$,
  $f(e_1)f(e_3)f(e_4) = \begin{pmatrix}
    -1 & 0 \\
    0 & 1
  \end{pmatrix}$,
  $f(e_3)f(e_2)f(e_4) = \begin{pmatrix}
    0 & 1 \\
    1 & 0
  \end{pmatrix}$
  であることを見れば, $f(e_1), f(e_2), f(e_3), f(e_4)$から$\mathbb{H}(2)$が生成されることは明らかであろう.
  (たとえば, $\begin{pmatrix}
    0 & i \\
    -i & 0
  \end{pmatrix} = \begin{pmatrix}
    0 & 1 \\
    1 & 0
  \end{pmatrix}\begin{pmatrix}
    -1 & 0 \\
    0 & 1
  \end{pmatrix}\begin{pmatrix}
    i & 0 \\
    0 & i
  \end{pmatrix}$など.)
\end{ans}

\begin{ans}
  計算するだけなので略.
\end{ans}

\end{document}
