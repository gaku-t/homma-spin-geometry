\documentclass[dvipdfmx]{amsart}

\usepackage{amsmath, amssymb}
\usepackage{tikz-cd}

\theoremstyle{definition}
\newtheorem{ans}{}
\numberwithin{ans}{section}

\newcommand{\fakesection}[1]{%
  \par\refstepcounter{section}% Increase section counter
  \sectionmark{#1}% Add section mark (header)
  \addcontentsline{toc}{section}{\protect\numberline{\thesection}#1}% Add section to ToC
  % Add more content here, if needed.
}

\newcommand{\Hom}[2]{\mathrm{Hom}(#1, #2)}


\begin{document}

\fakesection{第1章 クリフォード代数}

\begin{ans}
  $((c_1v_1 + c_2v_2) \otimes w)(B)
  = B(c_1v_1 + c_2v_2, w)
  = c_1B(v_1, w) + c_2B(v_2, w)
  = c_1(v_1 \otimes w)(B) + c_2(v_2 \otimes w)(B)
  = (c_1(v_1 \otimes w) + c_2(v_2 \otimes w))(B)
  $.
  もう一方の式も同様.
\end{ans}

\begin{ans}
  $F \times G(c_1v_1 + c_2v_2, w)
  = F(c_1v_1 + c_2v_2) \otimes G(w)
  = (c_1F(v_1) + c_2F(v_2)) \otimes G(w)
  = c_1F(v_1) \otimes G(w) + c_2F(v_2) \otimes G(w)
  = c_1F \times G(v_1, w) + c_2F \times G(v_2, w)
  $. よって第1引数について線形である. 第2引数についての線形性も同様に示せる.
\end{ans}

\begin{ans}
  写像$B: V^\ast \times W \rightarrow \Hom{V}{W}$を,
  $(v^\ast, w) \mapsto (\varphi: v \mapsto v^\ast(v)w)$と定義する.
  これは双線形なので, 次の図式を可換とするような線形写像$\beta$が存在する.
  \[
    \begin{tikzcd}
      V^\ast \times W \ar[r, "\otimes"] \arrow[rd, "B"'] & V^\ast \otimes W \ar[d, "\beta"] \\
      & \Hom{V}{W}
    \end{tikzcd}
  \]
  $\beta$が同型であることを示したいが,
  $V^\ast \otimes W$と$\Hom{V}{W}$はともに次元が$\dim V \cdot \dim W$なので,
  $\beta$が全射であることを見ればよい.
  そのためには, $V$の基底$\{e_1,.., e_n\}$と$W$の基底$\{f_1,..., f_m\}$を適当にとり,
  $1 \le \forall i \le n$, $1 \le \forall j \le m$について$\varphi_{ij}(e_k) = \delta_{ik}f_j\ (k = 1,..., n)$
  をみたすような$\varphi_{ij}$ ($\Hom{V}{W}$の基底) が$\beta$の像に入っていることをみればよい.
  このことは, $V^\ast$の双対基底$\{e^\ast_1,..., e^\ast_n\}$について
  $B(e^\ast_i, f_j)(e_k) = e^\ast_i(e_k)f_j = \delta_{ik}f_j = \varphi_{ij}(e_k)$であることから分かる.
\end{ans}

\begin{ans}
  (1) $(\bigwedge^p(V))^\ast \ni \alpha$に対して,
  $\alpha \circ \pi \circ \otimes^p$は$p$次交代形式である.
  逆に, $p$次交代形式$A$に対して, $\alpha \in (\bigwedge^p(V))^\ast$を,
  $\alpha(v_1 \wedge \cdots \wedge v_p) = A(v_1,... v_p)$を線形に拡張することで定義できる.
  この$\alpha$がwell-definedであることを確かめる必要があるが,
  まず$A$が$p$次線形形式であることから, $\bigotimes^pV$上のある線形形式$\beta$により
  $\beta(v_1 \otimes \cdots \otimes v_p) = A(v_1,... v_p)$と書ける.
  $A$の交代性から, $\beta$は$I(V) \cap \bigotimes^pV$上で$0$である. したがって, $\beta$は$\alpha$を誘導する.
  このように定義した$\alpha$に対して$\alpha \circ \pi \circ \otimes^p = A$が成り立つから,
  この対応により$(\bigwedge^p(V))^\ast$と$p$次交代形式全体を同一視できる.\\
  (2) 前小問で見たことと同様であるが, $A$は$p$次線形写像なので,
  線形写像$\beta: \bigotimes^pV \rightarrow U$で$A = \beta \circ \otimes^p$を満たすものがただ$1$つ存在する.
  $A$の交代性から$\beta$は$I(V) \cap \bigotimes^pV$上で$0$なので,
  $\alpha: \bigwedge^pV \rightarrow U$を誘導し, $\alpha \circ \wedge^p = A$である.
  $\beta$の一意性から, これを満たす$\alpha$は一意である.
  \[
    \begin{tikzcd}
      V \times \cdots \times V \ar[r, "\otimes^p"] \ar[rd, "A"'] &\bigotimes^pV \ar[d, "\beta"'] \ar[r, "\pi"] & \bigwedge^pV \ar[ld, dashed, "\alpha"]\\
      & U &
    \end{tikzcd}
  \]
  \\
  (3) テンソル代数の普遍性から, 代数準同型$F_T: T^\ast(V) \rightarrow A$で$F_T|_V = f$となるものがただ$1$つ存在する.
  $F_T$はイデアル$I(V)$上で$0$なので, 準同型$F: \bigwedge^\ast(V) \rightarrow A$を誘導する.
  $F$の一意性は, $F_T$の一意性からしたがう.
\end{ans}

\begin{ans}
  基底であることはすでに示してあったので, 内積を調べる.
  $\langle e_{i_1} \wedge \cdots \wedge e_{i_p}, e_{i_1} \wedge \cdots \wedge e_{i_p} \rangle
  = \det(\delta_{ij})_{ij} = 1$.
  また, $e_{i_1} \wedge \cdots \wedge e_{i_p} \neq e_{i^\prime_1} \wedge \cdots \wedge e_{i^\prime_p}\ (1 \le i_1 < \cdots < i_p \le n,\ 1 \le i^\prime_1 < \cdots < i^\prime_p \le n)$
  ならば, ある$i_j$について$i_j \notin \{i^\prime_1,..., i^\prime_p\}$である.
  よって行列$(\langle e_{i_k}, e_{i^\prime_l} \rangle)_{kl}$は第$j$行の成分がすべて$0$となり, 行列式は$0$である. よって,
  $\langle e_{i_1} \wedge \cdots \wedge e_{i_p} , e_{i^\prime_1} \wedge \cdots \wedge e_{i^\prime_p} \rangle = 0$.
\end{ans}

\end{document}
